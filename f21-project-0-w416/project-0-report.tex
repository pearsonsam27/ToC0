\documentclass[12pt,letterpaper]{ntdhw}



\title{Project 0: Functional Programming and Lisp Introduction}
\author{CSCI 561}

\rhead{Names:}

%\keytrue

\begin{document}
\pagestyle{fancyplain}

\maketitle
\thispagestyle{fancyplain}
%\clearpage

\begin{enumerate}

  \item What are the result types of the following Lisp expressions?
  \begin{enumerate}
    \item {\tt 1} : \emph{
      % Your Answer Here
    }
    \item {\tt (+ 1 2)} : \emph{
      % Your Answer Here
    }
    \item {\tt '(+ 1 2)} : \emph{
      % Your Answer Here
    }
    \item {\tt (eval '(+ 1 2))} : \emph{
      % Your Answer Here
    }
    \item {\tt (lambda () (+ 1 2))} : \emph{
      % Your Answer Here
    }
    \item {\tt "foo"} : \emph{
      % Your Answer Here
    }
    \item {\tt 'bar} : \emph{
      % Your Answer Here
    }
  \end{enumerate}

  \item Tail Calls:
  \begin{enumerate}
    \item What is tail recursion?

    \begin{emph}
      Answer: % Your Answer Here
    \end{emph}

    \item In the recursive implementation, will {\tt fold-left} or
    {\tt fold-right} be more memory-efficient?  Why?

    \begin{emph}
      Answer: % Your Answer Here
    \end{emph}
  \end{enumerate}

  \item Lisp and Python represent code differently.

  \begin{enumerate}
    \item Contrast the representations of Lisp code and Python code.

    \begin{emph}
      Answer: % Your Answer Here
    \end{emph}

    \item How do Python's {\tt eval()} and {\tt exec()} differ from
    the approach of Lisp?

    \begin{emph}
      Answer: % Your Answer Here
    \end{emph}
  \end{enumerate}

  \item GCC supports an extension to the C language that allows
  local/nested functions (functions contained in other functions).  A
  GCC local function can access local variables from its parent
  function.
  \begin{enumerate}
    \item What problems could arise if you return a function pointer
    to a GCC local function? \emph{(Hint: ``Funarg problem'')}

    \begin{emph}
      Answer: % Your Answer Here
    \end{emph}

    \item How does Lisp handle this problem?

    \begin{emph}
      Answer: % Your Answer Here
    \end{emph}

  \end{enumerate}

\end{enumerate}

\end{document}

%%% Local Variables:
%%% mode: latex
%%% TeX-master: t
%%% End:
